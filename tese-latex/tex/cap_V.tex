\chapter{Considera\c{c}\~oes finais}

Neste t\'opico est\~ao as \'ultimas explica\c{c}\~oes deste trabalho, quais as principais dificuldades, o que pode ser conclu\'ido e o que ele pode gerar de produtivo para startups, academia e empreendedores.

\section{Limita\c{c}\~oes e dificuldades encontradas}

A maior dificuldade encontrada durante \`a realiza\c{c}\~ao do trabalho foi na aplica\c{c}\~ao do question\'ario de perguntas aos empreendedores do grupo \emph{StartupGO}, apesar de contar com mais de 1000 membros nem todos s\~ao empreendedores ou possuem uma \emph{startup} em opera\c{c}\~ao no mercado, foi preciso ent\~ao fazer um levantamento das \emph{startups} cadastradas no grupo e abordar de forma direta seus propriet\'arios atrav\'es do \emph{chat}. Gra\c{c}as \`a ajuda do administrador da comunidade (Paolo Petrelli) foi poss\'ivel ent\~ao aplicar a pesquisa ao universo de 100 empresas.

A bibliografia que descreve a hist\'oria por tr\'as do nascimento do termo \emph{startup} tamb\'em se revelou um grande desafio, uma vez que a maioria do material encontrado n\~ao distinguia com clareza o per\'iodo do surgimento das \emph{startups} e empresas de capital de risco.

\section{Conclus\~ao}

Este trabalho apresentou um estudo importante quanto \`a ado\c{c}\~ao do MVP como ferramenta de aux\'ilio \`a r\'apida coleta de \emph{feedback} das hip\'oteses de mercado que uma \emph{startup} faz a respeito de seus clientes e apresentou o resultado deste estudo aplicado ao contexto das \emph{startups} inseridas no mercado goiano.

Como forma de acelerar a constru\c{c}\~ao de MVP's na internet, a \emph{GEM} Perpetuus foi constru\'ida a fim de que desenvolvedores \emph{web} adeptos do \emph{framework} Ruby on Rails pudessem modificar rapidamente o MVP a partir do \emph{feedback} de seus clientes, atrav\'es do uso de "entrega cont\'inua" gra\c{c}as \`a disponibiliza\c{c}\~ao de um ambiente de automatiza\c{c}\~ao de tarefas no fluxo de desenvolvimento.

O modelo de previs\~ao elaborado mostrou--se ineficiente na tarefa de prever os fatores que influenciam a probabilidade de uma \emph{startup} aderir ao MVP na constru\c{c}\~ao de seu primeiro produto.

\section{Trabalhos futuros}

Futuramente a \emph{gem} Perpetuus poderia suportar outros \emph{frameworks} de desenvolvimento \emph{web}, como \emph{Sinatra} e o \emph{Java Server Faces} dado que o cerne do \emph{plugin} pode gerar \emph{templates} para diferentes linguagens de programa\c{c}\~ao.

Pesquisas futuras podem verificar se a ferramenta desenvolvida de fato acelera a constru\c{c}\~ao de MVP's e se os \emph{templates} gerados pelo \emph{plugin} precisam oferecer suporte \`a outras tecnologias.

Outros modelos causais poderiam ser elaborados para procurar relacionar a quest\~ao de como Startups Enxutas s\~ao criadas e como o MVP \'e aplicado na concep\c{c}\~ao de produtos destas \emph{startups}.

Os resultados da pesquisa aplicada \`as \emph{startups} goianas podem fomentar novos trabalhos relacionados ao empreendedorismo no estado de Goi\'as, \'e desej\'avel tamb\'em que a expans\~ao da amostra de dados.