\chapter{Uma \emph{GEM} para construir MVP's}
\label{cap:gem}

% - - - - - - - - - - - - - - - - - - - - - - - - - - - - - - - - - - -
\section{Tecnologias Empregadas}

\subsection{Ruby}

Ruby \'e uma linguagem de programa\c{c}\~ao de script interpretada, criada em 1994 por Yukihiro Matsumoto, com grande inspira\c{c}\~ao nas linguagens Python e Perl. A linguagem possui a caracter\'istica de ser totalmente orientada a objeto, c\'odigo aberto e com tipagem din\^amica e forte.

Existe uma filosofia por tr\'as do Ruby, a linguagem foi desenvolvida objetivando as pessoas, buscando uma maior produtividade e fornecendo uma sintaxe muito limpa e elegante. Ruby foi projetada com o princ\'ipio da menor surpresa, tentando diminuir a frustra\c{c}\~ao durante a programa\c{c}\~ao. Seu criador tinha o objetivo de fazer uma linguagem que proporcionasse divers\~ao ao programador, diminuindo as dificuldades no desenvolvimento \cite{flanagan2008ruby}.

\subsection{Ruby on Rails}

Antes de descrever sobre o Ruby on Rails ser\'a necess\'ario entender o conceito de framework. Segundo \cite{hartl2012ruby}, framework \'e uma solu\c{c}\~ao para um conjunto de problemas em comum com o uso de classes e interfaces que disponibilizam objetos com funcionalidades comuns a v\'arias aplica\c{c}\~oes. A utiliza\c{c}\~ao de framework pode trazer benef\'icios em rela\c{c}\~ao \`a agilidade de desenvolvimento, podendo reduzir seus custos.

Com o Ruby on Rails pode--se desenvolver sistemas para a web utilizando a orienta\c{c}\~ao a objeto e com a estrutura MVC (Modelo, Visualiza\c{c}\~ao, Controle) para constru\c{c}\~ao em camadas. O framework possui alguns princ\'ipios baseados no desenvolvimento \'agil, com o conceito do Do not Repeat Yourself (DRY, N\~ao se repita), ou seja, n\~ao repetir c\'odigo, e Convention over Configuration (CoC, Conven\c{c}\~ao sobre Configura\c{c}\~ao) \cite{akita2006repensando}.

\subsection{RSpec}

oi eu escrevi aqui com acentos

\subsection{GitHub}

\subsection{Heroku}






