\chapter{Perpetuus, uma \emph{GEM} para construir MVP's}
\label{cap:gem}

% - - - - - - - - - - - - - - - - - - - - - - - - - - - - - - - - - - -
\section{Descri\c{c}\~ao da GEM}

A \emph{gem} Perpetuus foi desenvolvida para 

\section{Requisitos de Sistema}

Para utilizar o plugin desenvolvido os seguintes requisitos de sistema precisam ser preenchidos:

\begin{itemize}
\item Sistema operacional Unix
\item Ruby 1.9 ou superior
\item Rubygems 2.0 ou superior
\item Cliente Git
\item Heroku Toolbelt
\item Bundler 1.3 ou superior
\end{itemize} 

\section{Tecnologias Empregadas}

\subsection{Ruby}

Ruby \'e uma linguagem de programa\c{c}\~ao de script interpretada, criada em 1994 por Yukihiro Matsumoto, com grande inspira\c{c}\~ao nas linguagens Python e Perl. A linguagem possui a caracter\'istica de ser totalmente orientada a objeto, c\'odigo aberto e com tipagem din\^amica e forte.

Existe uma filosofia por tr\'as do Ruby, a linguagem foi desenvolvida objetivando as pessoas, buscando uma maior produtividade e fornecendo uma sintaxe muito limpa e elegante. Ruby foi projetada com o princ\'ipio da menor surpresa, tentando diminuir a frustra\c{c}\~ao durante a programa\c{c}\~ao. Seu criador tinha o objetivo de fazer uma linguagem que proporcionasse divers\~ao ao programador, diminuindo as dificuldades no desenvolvimento \cite{flanagan2008ruby}.

\subsection{Ruby on Rails}

Antes de descrever sobre o Ruby on Rails ser\'a necess\'ario entender o conceito de framework. Segundo \cite{hartl2012ruby}, framework \'e uma solu\c{c}\~ao para um conjunto de problemas em comum com o uso de classes e interfaces que disponibilizam objetos com funcionalidades comuns a v\'arias aplica\c{c}\~oes. A utiliza\c{c}\~ao de framework pode trazer benef\'icios em rela\c{c}\~ao \`a agilidade de desenvolvimento, podendo reduzir seus custos.

Com o Ruby on Rails pode--se desenvolver sistemas para a web utilizando a orienta\c{c}\~ao a objeto e com a estrutura MVC (Modelo, Visualiza\c{c}\~ao, Controle) para constru\c{c}\~ao em camadas. O framework possui alguns princ\'ipios baseados no desenvolvimento \'agil, com o conceito do Do not Repeat Yourself (DRY, N\~ao se repita), ou seja, n\~ao repetir c\'odigo, e Convention over Configuration (CoC, Conven\c{c}\~ao sobre Configura\c{c}\~ao) \cite{akita2006repensando}.

\subsection{RSpec}

RSpec \'e um \emph{framework} de testes escrito em Ruby, permitindo que se descrevam aplica\c{c}\~oes em linguagem de dom\'inio espec\'ifico em ingl\^es domain--specific language (DSL) \cite{chelimsky2010rspec}

Como a maioria dos projetos falha em manter uma documenta\c{c}\~ao atualizada, utilizar o RSpec pode ser uma excelente maneira de se documentar um projeto, pois o resultado dos exemplos executados, quando bem escritos, formam uma especificac\c{c}\~ao do projeto.

\subsection{GitHub}

O GitHub \'e um servi\c{c}o de hospedagem para projetos de desenvolvimento de \emph{software} que usam o sistema de controle de vers\~ao Git. GitHub oferece planos pagos para reposit\'orios privados e contas livres para projetos de c\'odigo aberto (open--source). Em maio de 2011, tornou--se o local mais popular para hospedagem de projetos de c\'odigo--fonte aberto \cite{loeliger2012version}

\subsection{Heroku}

Heroku \'e uma plataforma de hospedagem de servi\c{c}os na nuvem que suporta linguagens de programa\c{c}\~ao como Ruby, Java, PHP e frameworks como Ruby on Rails \cite{kemp2013professional}.

O diferencial do Heroku est\'a na automa\c{c}\~ao da configura\c{c}\~ao necess\'aria \`a cria\c{c}\~ao de um ambiente para colocar uma aplica\c{c}\~ao em funcionamento, assim programadores podem focar seus esfor\c{c}os no desenvolvimento do \emph{software}, ficando a cargo do Heroku juntar os m\'odulos desenvolvidos, resolver quest\~oes de depend\^encia e tornar o que antes era c\'odigo em uma aplica\c{c}\~ao dispon\'ivel aos clientes.

\section{Funcionamento da GEM}






