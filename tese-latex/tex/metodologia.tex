\chapter{Metodologia}
\label{cap:metodologia}

\section{Classifica\c{c}\~ao da Pesquisa}

\subsection{Pesquisa Descritiva}

A pesquisa descritiva objetiva conhecer e interpretar a realidade sem nela interferir para modific\'a--la \cite{churchill2009marketing}. Muitas das pesquisas de \emph{marketing} realizadas s\~ao de car\'ater conclusivo descritivo \cite{perin2000pesquisa}. Pode--se dizer que ela est\'a interessada em descobrir e observar fen\^omenos, procurando descrev\^e--los, classific\'a--los e interpret\'a--los. Al\'em disso, ela pode se interessar pelas rela\c{c}\~oes entre vari\'aveis e, desta forma, aproximar--se das pesquisas experimentais. A pesquisa descritiva exp\~oe as caracter\'isticas de determinada popula\c{c}\~ao ou de determinado fen\^omeno, mas n\~ao tem o compromisso de explicar os fen\^omenos que descreve, embora sirva de base para tal explica\c{c}\~ao. Normalmente ela se baseia em amostras grandes e representativas. 

O formato b\'asico de trabalho \'e o levantamento (\emph{survey}). Os estudos mais utilizados nesse tipo de pesquisa s\~ao: o longitudinal (coleta de informa\c{c}\~oes ao longo do tempo) e o transversal (coleta de informa\c{c}\~oes somente uma vez no tempo). As pesquisas descritivas compreendem grande n\'umero de m\'etodos de coleta de dados, os quais compreendem: entrevistas pessoais, entrevistas por telefone, question\'arios pelo correio, question\'arios pessoais e observa\c{c}\~ao.

\subsection{Pesquisa Causal}

Neste tipo de investiga\c{c}\~ao manipula--se deliberadamente algum aspecto da realidade. \'E usada para obter evid\^encias de rela\c{c}\~oes de causa e efeito. A causalidade pode ser inferida quando entre duas ou mais vari\'aveis houver varia\c{c}\~ao concomitante, ordem de ocorr\^encia correta das vari\'aveis no tempo e quando os outros poss\'iveis fatores causais forem eliminados \cite{mattar1996pesquisa}.

A pesquisa causal pretende mostrar de que modo ou por que o fen\^omeno \'e produzido. A formata\c{c}\~ao padr\~ao desse tipo de trabalho \'e a experimental. Um experimento \'e um projeto de pesquisa que envolve a manipula\c{c}\~ao de uma ou mais vari\'veis enquanto outras s\~ao mantidas constantes, e a medi\c{c}\~ao dos resultados \cite{churchill2009marketing} \cite{malhotra2006pesquisa}. Os gerentes de marketing tomam continuamente decis\~oes baseadas em rela\c{c}\~oes causais presumidas.

\section{Operacionaliza\c{c}\~ao do M\'etodo}

Para alcan\c{c}ar o objetivo estabelecido elaborou--se um question\'ario com dez quest\~oes na tentativa de identificar o perfil das \emph{startups} adeptas do MVP.

O question\'ario elaborado foi validado junto a dois pesquisadores da \'area de \emph{marketing}, os professores Marcos In\'acio Severo de Almeida e Ricardo Limongi Fran\c{c}a Coelho, ambos professores da Faculdade de Aministra\c{c}\~ao e Ci\^encias Cont\'abeis da Universidade Federal de Goi\'as.

Posteriormente aplicou--se o question\'ario no grupo de \emph{startups} participantes do StartupGO no \emph{facebook} (https://www.facebook.com/groups/startupgo/). Com a ajuda do administrador da p\'agina Paolo Petrelli foram levantadas 100 \emph{startups} dispostas \`a participar da pesquisa, apesar do StartupGO contar atualmente com 1.047 membros a maioria dos participantes n\~ao s\~ao empreendedores.

As perguntas elaboradas e as respostas s\~ao mostradas na sess\~ao de An\'alise de Resultados. Todos os dados foram analisados utilizando--se a vers\~ao 21.0 do IBM SPSS (http://www.ibm.com/SPSS)

Com o intuito de prop\^or uma alternativa \`a r\'apida constru\c{c}\~ao de produtos m\'inimos vi\'aveis, produziu--se a \emph{gem} Perpetuus, utilizando linguagem de programa\c{c}\~ao Ruby, versionando o c\'odigo produzido em reposit\'orio online no endere\c{c}o: https://github.com/ricardobond/perpetuus/ do servi\c{c}o de hospedagem de c\'odigo oferecido pelo GitHub.