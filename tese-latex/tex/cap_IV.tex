\chapter{OI EU SOU UM TESTE}
\label{cap:questionario}

\section{TESTE}

apenas um teste com \ref{tab:pergunta1}

\begin{table*}[hb]
\centering
\caption{Tempo de opera\c{c}\~ao da \emph{startup} no mercado}
\label{tab:pergunta1}
\begin{tabular}{|p{10cm}|p{2cm}|}
\hline{\bf H\'a quanto tempo sua startup est\'a operando no mercado?} & {\bf Frequ\^encia}\\
\hline Menos de 6 meses & 42\\
\hline De 6 meses a 2 anos & 42\\
\hline Mais de 2 anos & 16\\
\hline TOTAL & 100\\
\hline
\end{tabular}
\end{table*}

\begin{table*}[hb]
\centering
\caption{Segmentos de mercado de opera\c{c}\~ao das \emph{startups}}
\label{tab:pergunta2}
\begin{tabular}{|p{10cm}|p{2cm}|}
\hline{\bf Em qual dos segmentos de mercado abaixo voc\^e enquadraria sua startup?} & {\bf Frequ\^encia}\\
\hline Minha startup est\'a operando em um mercado j\'a existente & 46\\
\hline Minha startup est\'a criando um mercado novo & 20\\
\hline Minha startup est\'a ressegmentando um mercado existente oferecendo um produto de custo inferior ao dos concorrentes & 23\\
\hline Minha startup est\'a ressegmentando um mercado existente oferecendo um produto de nicho de custo superior ao dos concorrentes & 8\\
\hline N\~ao consigo definir qual o mercado da minha startup & 3\\
\hline TOTAL & 100\\
\hline
\end{tabular}
\end{table*}

\begin{table*}[hb]
\centering
\caption{Planejamento feito para viabilizar a execu\c{c}\~ao da ideia}
\label{tab:pergunta3}
\begin{tabular}{|p{10cm}|p{2cm}|}
\hline{\bf Para colocar a sua ideia em pr\'atica, que tipo de planejamento voc\^e fez?} & {\bf Frequ\^encia}\\
\hline Business Canvas & 43\\
\hline Plano de Neg\'ocio & 28\\
\hline Lean Canvas & 9\\
\hline Outro & 20\\
\hline TOTAL & 100\\
\hline
\end{tabular}
\end{table*}

\begin{table*}[hb]
\centering
\caption{Tipos de recurso financeiro utilizados pelas \emph{startups}}
\label{tab:pergunta4}
\begin{tabular}{|p{10cm}|p{2cm}|}
\hline{\bf Que tipo de recurso financeiro voc\^e utilizou para construir a primeira vers\~ao do produto da sua startup?} & {\bf Frequ\^encia}\\
\hline Recurso pr\'oprio & 70\\
\hline Recurso de amigos, fam\'ilia ou colegas & 22\\
\hline Recurso de um investidor de renome no mercado & 4\\
\hline Recurso de uma empresa de capital de risco & 4\\
\hline TOTAL & 100\\
\hline
\end{tabular}
\end{table*}

\begin{table*}[hb]
\centering
\caption{Investimento necess\'ario \`a constru\c{c}\~ao da primeira vers\~ao do produto}
\label{tab:pergunta5}
\begin{tabular}{|p{10cm}|p{2cm}|}
\hline{\bf Quanto foi investido na primeira vers\~ao do produto?} & {\bf Frequ\^encia}\\
\hline Menos de R\$ 500,00 & 19\\
\hline De R\$ 500,00 a R\$ 2.000,00 & 22\\
\hline De R\$ 2.000,00 a R\$ 10.000,00 & 23\\
\hline Mais de R\$ 10.000,00 & 36\\
\hline TOTAL & 100\\
\hline
\end{tabular}
\end{table*}

\begin{table*}[hb]
\centering
\caption{Tempo gasto para construir a primeira vers\~ao do produto}
\label{tab:pergunta6}
\begin{tabular}{|p{10cm}|p{2cm}|}
\hline{\bf Quanto tempo voc\^e levou para colocar a primeira vers\~ao do produto/servi\c{c}o de sua startup no ar?} & {\bf Frequ\^encia}\\
\hline Menos de 1 semana & 4\\
\hline De 1 a 3 semanas & 10\\
\hline De 1 a 3 meses & 28\\
\hline De 3 a 6 meses & 24\\
\hline Mais de 6 meses & 34\\
\hline TOTAL & 100\\
\hline
\end{tabular}
\end{table*}

\begin{table*}[hb]
\centering
\caption{Eu sou a legenda}
\label{tab:pergunta7}
\begin{tabular}{|p{10cm}|p{2cm}|}
\hline{\bf Ap\'os definir as funcionalidades ou caracter\'isticas de seu produto/servi\c{c}o, quanto do que foi definido estava presente na primeira vers\~ao comercializada?} & {\bf Frequ\^encia}\\
\hline Menos da metade & 39\\
\hline Metade (pode ser aproximado) & 25\\
\hline Mais da metade & 25\\
\hline Tudo & 11\\
\hline TOTAL & 100\\
\hline
\end{tabular}
\end{table*}

\begin{table*}[hb]
\centering
\caption{Prioridade de implementa\c{c}\~ao de funcionalidades}
\label{tab:pergunta8}
\begin{tabular}{|p{10cm}|p{2cm}|}
\hline{\bf Qual dos itens abaixo foi implementado com maior prioridade na primeira vers\~ao de seu produto?} & {\bf Frequ\^encia}\\
\hline Funcionalidades que os clientes julgam \'uteis & 50\\
\hline Funcionalidades que voc\^e julga \'uteis ao cliente & 34\\
\hline Campanhas de marketing (ex.: Google ad--words) & 0\\
\hline Nenhuma das anteriores & 16\\
\hline TOTAL & 100\\
\hline
\end{tabular}
\end{table*}

\begin{table*}[hb]
\centering
\caption{Empreendedores que fazem uso do MVP}
\label{tab:pergunta9}
\begin{tabular}{|p{10cm}|p{2cm}|}
\hline{\bf Partindo do princ\'ipio de que um MVP (Produto M\'inimo Vi\'avel) \'e o m\'inimo conjunto de funcionalidades que permite uma a\c{c}\~ao e aprendizado sobre os clientes ou usu\'arios. Voc\^e faz/fez uso do MVP  em sua startup?} & {\bf Frequ\^encia}\\
\hline Sim & 67\\
\hline N\~ao & 33\\
\hline TOTAL & 100\\
\hline
\end{tabular}
\end{table*}

\begin{table*}[hb]
\centering
\caption{Tipos de MVP mais utilizados pelos empreendedores consultados}
\label{tab:pergunta10}
\begin{tabular}{|p{10cm}|p{2cm}|}
\hline{\bf Caso tenha respondido ``N\~ao`` \`a pergunta anterior pule esta pergunta, caso tenha respondido ``Sim`` selecione abaixo os tipos de MVP que voc\^e utilizou para vender seu produto/servi\c{c}o. (Marque todos que j\'a tenha usado)} & {\bf Frequ\^encia}\\
\hline Apresenta\c{c}\~ao de Slides & 12\\
\hline P\'agina de pr\'e--lan\c{c}amento + Formul\'ario de Inscri\c{c}\~ao + Adwords & 19\\
\hline Prot\'otipo & 37\\
\hline V\'ideo & 15\\
\hline Trabalho manual & 6\\
\hline Outro & 20\\
\hline TOTAL & 109\\
\hline
\end{tabular}
\end{table*}