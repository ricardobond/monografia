\keys{Lean Startup, MVP, Continuous Delivery, Ruby on Rails}

\begin{abstract}{The MVP's adoption as a tool for validating market assumptions for lean startups}
This paper describes a study whose overall objective is to identify characteristics of startups who use the MVP (Minimum Viable Product) as a tool for validation of market assumptions made by its entrepreneurs. To achieve the proposed objective a survey was conducted with 100 startups from Goi\^ania's group members StartupGO. Parallel built a plugin for developers using Ruby on Rails framework can accelerate the process of building MVP's using the methodology of continuous delivery. In the analysis of results from the questionnaire were conducted descriptive and inferential statistics, descriptive statistics analysis verified the frequency of responses and were relizados correlation tests and chi--square between the variables, inferential analysis considered the application of the model logistic regression. The plugin designed accelerates the development of MVP's for automating testing tasks, build and deploy the code produced by the team of web development startup by placing code in produçãi whenever he successfully pass the tasks mentioned, applying the concept of continuous delivery.
\end{abstract}
