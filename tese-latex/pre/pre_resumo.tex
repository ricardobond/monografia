\chaves{Startup Enxuta, MVP, Entrega Cont\'inua, Ruby on Rails.}

\begin{resumo} 

Este trabalho apresenta uma pesquisa cujo objetivo geral \'e identificar caracter\'isticas de \emph{startups} que utilizam o MVP (Produto M\'inimo Vi\'avel) como ferramenta de valida\c{c}\~ao das hip\'oteses de mercado elaboradas por seus empreendedores.

Para alcan\c{c}ar o objetivo proposto foi realizada uma pesquisa com 100 \emph{startups} goianas integrantes do grupo StartupGO.

Paralelamente foi desenvolvido um \emph{plugin} para que desenvolvedores que utilizam o \emph{framework} Ruby on Rails possam acelerar o processo de constru\c{c}\~ao de MVP's utilizando a metodologia de entrega cont\'inua.

Na an\'alise de resultados provenientes da aplica\c{c}\~ao do question\'ario foram realizadas estat\'isticas descritivas e inferenciais, na an\'alise da estat\'istica descritiva verificou--se a frequ\^encia das respostas e foram relizados testes de correla\c{c}\~ao e qui--quadrado entre as vari\'aveis, a an\'alise inferencial considerou a aplica\c{c}\~ao do modelo de regress\~ao log\'istica.

O \emph{plugin} concebido acelera o desenvolvimento de MVP's por automatizar tarefas de teste, \emph{build} e \emph{deploy} do c\'odigo produzido pela equipe de desenvolvimento \emph{web} da \emph{startup}, colocando o c\'odigo em produ\c{c}\~ao sempre que ele passar com sucesso pelas tarefas citadas, aplicando--se o conceito de entrega cont\'inua.

\end{resumo}

