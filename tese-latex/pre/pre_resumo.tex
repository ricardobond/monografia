\chaves{Startup Enxuta, MVP, Entrega Cont\'inua, Ruby on Rails.}

\begin{resumo} 
\emph{Startups} s\~ao concebidas com o objetivo de desenvolver um modelo de neg\'ocio repetit\'ivel e escal\'avel, entretanto os empreendedores devem ter em mente que a fase inicial de uma \emph{startup} \'e marcada por um cen\'ario de extrema incerteza. \'E consenso que estamos vivendo um renascimento do empreendedorismo mundial, mas a car\^encia de um paradigma gerencial que permita a r\'apida valida\c{c}\~ao do feedback de demandas do usu\'ario faz com que oportunidades e iniciativas de inova\c{c}\~ao corram riscos altos. O que torna o fracasso especialmente doloroso n\~ao \'e apenas o dano econ\^omico causado a funcion\'arios, empresas e investidores; eles tamb\'em s\~ao um desperd\'icio gigantesco dos recursos mais preciosos da nossa civiliza\c{c}\~ao: o tempo, a paix\~ao e a habilidade das pessoas. O movimento da \emph{Startup Enxuta} dedica--se \`a impedir que o insucesso fruste novos empreendedores, para tal o produto m\'inimo vi\'avel (MVP) \'e apresentado como ferramenta de aux\'ilio na r\'apida valida\c{c}\~ao de hip\'oteses de mercado. Este trabalho prop\~oe al\'em de um estudo sobre o perfil de 100 \emph{startups} goianas focado no uso do MVP, um plugin para desenvolvedores que trabalham com \emph{Ruby on Rails} permitindo a constru\c{c}\~ao de MVP's de maneira mais r\'apida a partir de um ambiente pr\'e-configurado com o foco na entrega cont\'inua.
\end{resumo}

